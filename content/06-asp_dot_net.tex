%! Licence = CC BY-NC-SA 4.0

%! Author = mariuszindel
%! Date = 30. Jan 2022
%! Project = Cheat-Sheet-Web-Engineering-3


\section{ASP .NET}
Enterprise Framework, Kompilierbare Sprache, Komplett neue Entwicklung, alle Betriebssystem

\subsection{Dependency Injection}
ASP.NET kommt mit primitiven DI Container.\\
\textbf{Idee:} Klasse erwähnt welche Interfaces benötigt werden. Ein Resolver sucht im Container nach einer geeigneten Klasse und übergibt diese.
%\begin{lstlisting}[style=CSharp]
%public class Startup {
% public void ConfigureServices(
%                       IserviceCollection services) {
%  services.AddTransient<IUserService, UserService>();
%  services.AddControllers();
%  services.AddMvc(); }
%public void Configure(IApplicationBuilder app,
%                           IHostingEnvironment env) {
% app.UseRouting();
% app.UseEndpoints(endpoints => {
%   endpoints.MapControllers();
%   endpoints.MapRazorPages();  }); } }
%\end{lstlisting}

\subsubsection{Lifetime}
\textbf{Transient:} jedes Mal erstellt wenn angefordert. besten geeignet für leichte, zustandslose Dienste.\\
\textbf{Scoped:} Erstellt 1x pro Request.\\
\textbf{Singleton:} Erstellt bei 1. mal requested. nachfolgende Anfragen dieselbe Instanz verwenden

\subsubsection{Captive Dependency}
Komponenten dürfen nur Komponenten mit gleicher oder längerer Lebensdauer Injecten lassen.

\subsection{Projekt-Struktur}
\textbf{wwwroot:} Statische Inhalte (CSS,JS,HTML)\\
\textbf{appsettings.json:} Einstellungen der Webseite (Connection-String für DB / AllowedHosts)\\
\textbf{Programm.cs:} Einstiegspunkt der WebApp\\
\textbf{Startup.cs:} Konfiguriert die WebApp

\subsection{Razor}
%\begin{lstlisting}[style=CSharp]
%@{  var name = "John Doe";
%    var weekDay = DateTime.Now.DayOfWeek; }
%<p>Hello @name, today is @weekDay</p>
%@foreach(var item of todos) {
%    <p>@item.text</p>
%}
%@if(/* ... */)
%\end{lstlisting}

\subsubsection{Wichtige Dateien}
\textbf{Shared/\_layout.cshtml:} Generelles layout der App. Definiert Sections (Placeholders)
\begin{lstlisting}[style=CSharp]
@RenderBody() // Platz für Content
@RenderSection("Nav", false); // Platz für Section
@section Nav{ /* ... */ }
\end{lstlisting}
\textbf{\_ViewStart.cshtml:} Hierarchisch, Code welcher vor den Razor-Files ausgeführt wird.
\textbf{\_ViewImports.cshtml:} Hierarchisch, Namespaces / Tag-Helpers können registriert werden.

%\subsubsection{Partials}
%Markup Files, verwendet innerhalb von anderen Markup Files.
%Bessere Aufteilbarkeit und Wiederverwendbarkeit.
%\begin{lstlisting}[style=CSharp]
%<partial name="_Card" for="Card1" />
%<partial name="_Card" model="..." />
%\end{lstlisting}

%\subsubsection{View Components}
%Mächtigere Variante von Partials.
%Beinhalten Logik, können Daten laden / auf bearbeiten.
%Rendert ein Teil der Webseite.\\
%\textbf{Unterschied zu Pages:} Rendern nur ein Teil der Seite.\\
%\textbf{Location:} \textit{/ViewComponents}\\
%Razor-File: \textit{Pages/Shared/Components/[ComponentName]/[ViewName]}
%\begin{lstlisting}[style=CSharp]
%public class ToDoList: ViewComponent {
%    public string[] Todos { get; set; }
%    public ToDoList() { /* ... */ }
%    public IViewComponentResult Invoke() {
%        // /Pages/Shared/Components/TodoList/Default
%        return View(Todos);
%    }
%}
%// Razor File
%@Page
%@{ ViewData["Title"] = "ViewComponent"; }
%<vc:to-do-list></vc:to-do-list>
%@await Component.InvokeAsync("ToDoList")
%\end{lstlisting}

\subsubsection{ViewData / TempData}
Mit Attribut Gekennzeichnete Daten werden allen Razor-Files im Render-Baum übergeben.\\
\textbf{ViewData/ViewBag:} Daten an Layout\\
\textbf{TempData:} Überlebt redirect, Middleware nötig

\subsection{Pages}
Alternative / vereinfachte Variante vom MVC. Router muss nicht konfiguriert. Best-Practices für Serverside-Rendering.
\textbf{Kombination MVC:} Static Seite mit Pages, REST-API mit MVC.

%\subsubsection{Routing}
%WebApp generiert anhand der URL eine Antwort.
%Bei einem Aufruf wird im Folder \textit{/pages/} nach einer Page gesucht und ausgeführt (\textbf{case insensitive}):\\
%\textit{/add} $\rightarrow$  \textit{/pages/add.cshtml}

\subsubsection{Model (*.cshtml.cs:)}
Pro HTTP-Verb kann im VM eine Funktion definiert werden, die davor aufgerufen wird
\begin{lstlisting}[style=CSharp]
public class HelloWorldModel : PageModel {
  public int Id { get; set; }
  public void OnPost(int id) {
      Id = id; } }
// oder ohne kopieren der Properties
public class HelloWorldModel : PageModel {
  [BindProperty(SupportsGet = true)]
  public int Id { get; set; } } }
\end{lstlisting}

\subsubsection{View (*.cshtml)}
\textbf{@page}: Definiert das Razor-File als Page\\
\textbf{@page ''/test'':} Überschreibt Default-Routing
\begin{lstlisting}[style=CSharp]
@page "/test/{id:int?}"
@model HelloWorldModel
@{ ViewData["Title"] = "HelloWorld"; }
<h1>@Model.HelloWorld</h1>
\end{lstlisting}

\subsection{AJAX}

\subsubsection{Handlers}
Pages können Actions als handler anbieten.\\
\textbf{Namenskonvention:} \textit{On[Method][Name]}
\begin{lstlisting}[style=CSharp]
public IActionResult OnPostEcho(strong echoText) {
    return this.Content(echoText); }
// Ajax Razor Form
<form asp-page="Ajax" asp-page-handler="Echo" data-ajax="true" data-ajax-method="POST" data-ajax-mode="replace" data-ajax-update="#output">
  <input name="echoText" /><input type="submit" />
</form>
\end{lstlisting}
\textbf{Zugriff:} \textit{[Method]: /[Page]?handler=[Handler]}\\
Bsp: POST auf \textit{/Ajax?handler=echo}
%\textbf{Rückgabewerte (IActionResult)}
%\begin{itemize}
%    \item ContentResult: String-, Json-, EmptyResult
%    \item Status: NotFoundResult, StatusCodeResult
%    \item Redirects: RedirectToPage, RedirectToPagePermanent
%    \item Hilfsmethoden: Page(), Partial(), Content()
%\end{itemize}

\subsection{Entity Framework}
\textbf{Code First benötigt:} Type Discovery (Welche Klassen in die DB), Connection String (Wohin mit Daten), DbContext (Entry Point)\\
\textbf{Migration:} EF Core erlauft keine automatische Migrationen von Model Änderungen mehr. Aktuell nur über Konsole: \textit{dotnet ef database update}

\subsubsection{Entity Konventionen}
\textbf{public [long/string] Id:} automatisch PK\\
\textbf{public virtual ApplicationUser Customer:} Wird als Navigation Property erkannt\\
\textbf{public [long/string] CustomerId:} Wird als Foreign Key für das Customer Property erkannt

\subsubsection{Wichtige Attribute}
\textbf{[Required]:} NotNull in DB,
\textbf{[NotMapped]:} Nicht in DB geschrieben,
\textbf{[Key]:} Definiert den PK,
\textbf{[MaxLength(10)]:} Allokationsgrösse in DB

\subsection{Validierung}

\subsubsection{Schritt 1: Annotieren der Klassen}
\textbf{Attribute:}
\texttt{\tiny [StringLength(60, MinimumLength=3)]},\\
\texttt{\tiny [RegularExpression(@"..")]},\texttt{\tiny [DataType(DataType.Date)]}
\texttt{\tiny [Required]}. Attribute \underline{sind kombinierbar}

\subsubsection{Schritt 2: Razor anpassen}
\textbf{Validation ins DOM einfügen:}
\begin{lstlisting}[style=CSharp]
<div asp-validation-summary="ModelOnly"></div>
<span asp-validation-for="Item.Name"></span>
\end{lstlisting}
\textbf{JQuery Validation einbinden:}
\begin{lstlisting}[style=CSharp]
@section Scripts {
    <script src=".../jquery.validate.js"></script> }
\end{lstlisting}

\subsubsection{Schritt 3: Serverseitige Validierung}
\begin{lstlisting}[style=CSharp]
[HttpPost]
public ActionResult Index(Order order) {
    if(ModelState.IsValid) {
        order.CustomerId = User.Identity.GetUserId();
        _db.Orders.Add(order); _db.SaveChanges();
        return View("OrderOk", order);
    } return BadRequest(); }
\end{lstlisting}

\subsection{Authentifizierung}
\textbf{ASP.NET Identity Features:} PW Stärke, User Validator, Lockout, 2Faktor, Reset, OAuth\\
\textbf{ASP.NET Identity Klassen:}\\
\texttt{\tiny UserManager<ApplicationUser>},\texttt{\tiny RoleManager<IdentityRole>}
\texttt{\tiny IAuthorizationService},
\texttt{\tiny SingInManager}

\subsubsection{Aktivierung \& Konfiguration (Startup.cs)}
\begin{lstlisting}[style=CSharp]
services.AddDefaultIdentity<IdentityUser>() // <- DI
    .AddEntityFrameworkStores<ApplicationDbContext>()
    .AddDefaultTokenProviders();
app.UseIdentity(); // <-Middleware | Einstellungen->
services.AddDefaultIdentity<IdentityUser>(options=>{
    options.Password.RequireDigit = false;
    options.Password.RequiredLength = 8;
}).AddRoles<IdentityRole>()
.AddEntityFrameworkStores<ApplicationDbContext>()
\end{lstlisting}

%\subsubsection{Anwenden}
%\textbf{Attribute:}\\
%\textit{[Authorize]:} User muss authentifiziert sein (Controller/Actions)\\
%\textit{[AllowAnonymous]:} Ausnahme für spezifische Action
%\begin{lstlisting}[style=CSharp]
%this.User // Eingeloggter User Typ: ClaimsPrincipal
%// CRUD Operationen über ApplicationUsers von DI
%var user = await _userManager.GetUserAsync(User);
%var id = _userManager.GetUserId(User);
%\end{lstlisting}
%\textbf{Claim:} Statement über einen User, ausgestellt von einem Identity Provider.\\

\subsubsection{Authentifizierung prüfen}
\begin{lstlisting}[style=CSharp]
[Authorize] // <-Automatisch
public ActionResult Create() { return View(order); }
public ActionResult Create() { // <- Manuell
  if(User.Identity.IsAuthenticated){ return View(..)}
  else { return new StatusCodeResult(401); } }
\end{lstlisting}

\subsection{Authorisierung}
\begin{lstlisting}[style=CSharp]
// Lösung 1: Attribute:
[Authorize(Roles = "Admin, PowerUser")]
[Authorize(Policy = "OlderThan18")]
// Lösung 2: Services:
isInRole=await userManager.IsInRoleAsync(user,"Admi")
// Lösung 3: Claims (z.B. Facebook)
User.HasClaim(ClaimTypes.Role, "Admin")
\end{lstlisting}

\subsubsection{Authorizierung mit Razor}
\begin{lstlisting}[style=CSharp]
@inject UserManager<ApplicationUser> manager;
@inject ApplicationDbContext context;
@{  var user = await manager.GetUserAsync(User);
 if(user != null && await manager.IsInRoleAsync(
    user, "Admin")){ /* ... */ } }
\end{lstlisting}

\subsection{Testing}
\begin{lstlisting}[style=CSharp]
public class UnitTest {
    [Fact]
    public void TestName() { /* ... */ }}
\end{lstlisting}
%
%\subsection{App Secrets}
%Ermöglicht es, Secrets in einem separaten File zu persistieren.
%

\subsection{API Routing}
\begin{lstlisting}[style=CSharp]
[Route("api/[controller]")]
public class ValuesController : Controller{
  [HttpGet]
  public IEnumerable<Value> Get(){
    return _valueService.All(); }
  [HttpGet("{id}")]
  public Value Get(int id) {
    return _valueService.Get(id); }
  [HttpPost]
  public void Post([FromBody]Value value) {
    _valueService.Add(value);}
\end{lstlisting}






\subsection{BMI Rechner}
\textbf{Startup.cs}:
\begin{lstlisting}[style=CSharp]
public class Startup {
  public Startup(IConfiguration configuration){
    Configuration = configuration; }
  public IConfiguration Configuration { get; }
  public void ConfigureServices(IServiceCollection services){
    services.AddMvc();
    services.AddSingleton<IBmiService, BmiService>(); }
  public void Configure(IApplicationBuilder app, IWebHostEnvironment env){
    app.UseStaticFiles();
    app.UseRouting();
    app.UseEndpoints(endpoints => {
      endpoints.MapRazorPages();}); } }
\end{lstlisting}

\textbf{Bmi.cshtml}:
\begin{lstlisting}[style=CSharp]
@page
@model BmiModel
@{
    ViewData["Title"] = "Bmi";
    Layout = null;

    double value = Model.BmiValue;



}


\end{lstlisting}




\textbf{Bmi.cshtml.cs}:
\begin{lstlisting}[style=CSharp]
public class BmiModel : PageModel {
  private readonly IBmiService _bmiService;
  public BmiModel(IBmiService bmiService) {
    _bmiService = bmiService; }
  [BindProperty(SupportsGet = true)]
  public Bmi Bmi { get; set; }
  public double BmiValue { get; set; }
  public void OnGet() {
    BmiValue = _bmiService.Calculcate(Bmi);}
  public void OnPost(Bmi bmi) {
    Bmi = bmi;
    BmiValue = _bmiService.Calculcate(Bmi);}}
\end{lstlisting}

\textbf{Other stuff:}
\begin{lstlisting}[style=CSharp]
public class Bmi {
  [Range(0, 300)]
  [Display(Name = "Gewicht in kg")]
  public double Weight { get; set; }

  [Range(30, 250)]
  [Display(Name = "Höhe in cm")]
  public double Height { get; set; }}

public interface IBmiService {
    double Calculcate(Bmi data); }

public class BmiService : IBmiService {
  public double Calculcate(Bmi data) {
    return Math.Round(data.Weight / Math.Pow((data.Height / 100), 2), 2); }}
\end{lstlisting}






\vfill
$ $
